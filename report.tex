\documentclass{article}
\usepackage{geometry}
\usepackage{booktabs}
\usepackage{hyperref}

\title{Performance Comparison: Li-Chao Tree vs. Dynamic Convex Hull Trick}
\author{chnlich}
\date{\today}

\begin{document}

\maketitle

\section{Introduction}
This report compares the performance of two popular data structures for maintaining the lower/upper convex hull of a set of lines and querying extremum values:
\begin{enumerate}
    \item \textbf{Li-Chao Tree}: A segment tree based approach where each node stores a line that dominates the center of the interval.
    \item \textbf{Dynamic Convex Hull Trick (CHT)}: A balanced binary search tree (std::set) approach maintaining the hull directly by storing lines sorted by slope and maintaining intersection points.
\end{enumerate}

\section{Implementation Details}
\subsection{Li-Chao Tree}
We implemented a sparse, pointer-based Li-Chao Tree supporting a coordinate range of $[-10^9, 10^9]$. The tree is built dynamically as lines are inserted.

\subsection{Dynamic CHT}
We used the standard method (often referred to as the "LineContainer" from KACTL) which stores lines in a `std::set`, sorted by slope. It maintains the validity of the hull by removing lines that are no longer part of the hull upon insertion.

\section{Experimental Setup}
The benchmarks were run on a standard environment with the following distributions:
\begin{itemize}
    \item \textbf{Random}: Lines ($k, m$) and queries ($x$) are generated uniformly at random.
    \item \textbf{Monotonic K}: Lines are inserted in increasing order of slope $k$.
    \item \textbf{Monotonic X}: Queries are performed in increasing order of $x$.
\end{itemize}

\section{Results}
Preliminary results (see `results.md` in repository) suggest that Dynamic CHT is generally faster than the pointer-based Li-Chao Tree for $N \le 200,000$, likely due to the overhead of dynamic node allocation in the Li-Chao Tree. However, Li-Chao Tree offers better worst-case guarantees for arbitrary updates compared to CHT which can degrade if line removals are required (though standard CHT only supports addition).

\section{Future Work}
\begin{itemize}
    \item Implement an implicit (array-based) Li-Chao Tree for smaller coordinate ranges to check if cache locality improves performance.
    \item Test on larger datasets ($N=10^6, 10^7$).
    \item Compare memory usage.
\end{itemize}

\end{document}
