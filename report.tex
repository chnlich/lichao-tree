\documentclass{article}
\usepackage{geometry}
\usepackage{booktabs}
\usepackage{hyperref}
\usepackage{amsmath}
\usepackage{graphicx}

\title{Li-Chao Tree Optimization Report (Iteration 5 - FINAL)}
\author{chnlich}
\date{\today}

\begin{document}

\maketitle

\section{Introduction}
This is the final iteration (v5) of the Li-Chao Tree improvement cycle. Having established efficient Standard, Static, Discrete, Iterative, and Segment variants, we now introduce the **Persistent Li-Chao Tree** and provide comprehensive documentation. This completes the suite, offering solutions for standard, memory-constrained, coordinate-compressed, segment-based, and history-dependent problems.

\section{Methodology}

\subsection{Persistent Li-Chao Tree}
The Persistent Li-Chao Tree (\texttt{lichao\_persistent.hpp}) allows for versioned updates. Unlike the standard variant which modifies the tree in-place, the persistent variant creates new nodes for the path affected by an update, preserving the previous version of the tree.
\begin{enumerate}
    \item \textbf{Structure}: An array-based node pool where each node stores indices to left/right children.
    \item \textbf{Update}: returns a new root index. Space complexity is $O(\log C)$ per update.
    \item \textbf{Query}: takes a root index and performs the standard Li-Chao query.
    \item \textbf{Use Case}: Essential for problems requiring queries on past states (e.g., "what was the minimum at time $t$?") or for tree path optimization problems where persistence allows querying from root to node.
\end{enumerate}

\subsection{Documentation}
A comprehensive \texttt{README.md} has been added to the repository, serving as a user guide to select the appropriate variant for any given problem.

\section{Verification}
We extended the verification suite in \texttt{main.cpp} to test the Persistent Li-Chao Tree.
\begin{itemize}
    \item \textbf{Test Case}: Created branching versions of the tree (Root 1, Root 2 based on Root 1, Root 3 based on Root 1).
    \item \textbf{Result}: Queries on different roots returned correct values independent of subsequent updates in other branches, confirming true persistence.
\end{itemize}

\section{Final Suite Summary}

The repository now contains:

\begin{itemize}
    \item \textbf{Standard} (\texttt{lichao.hpp}): Dynamic pointer-based.
    \item \textbf{Static} (\texttt{lichao\_static.hpp}): Fixed array-based.
    \item \textbf{Discrete} (\texttt{lichao\_discrete.hpp}): Coordinate compressed.
    \item \textbf{Iterative} (\texttt{lichao\_iterative.hpp}): Non-recursive.
    \item \textbf{Segment} (\texttt{lichao\_segment.hpp}): For line segments.
    \item \textbf{Persistent} (\texttt{lichao\_persistent.hpp}): For versioned queries.
\end{itemize}

\section{References}

\begin{enumerate}
    \item **Li Chao**, "The Application of Segment Tree in Geometry Problems", \textit{Winter Camp (WC) 2012}, China. (Original proposal of the data structure).
    \item **CP-Algorithms**, "Li Chao Tree", \url{https://cp-algorithms.com/geometry/li_chao_tree.html}.
    \item **Codeforces**, "Li Chao Tree Tutorial", \url{https://codeforces.com/blog/entry/51275}.
\end{enumerate}

\end{document}
